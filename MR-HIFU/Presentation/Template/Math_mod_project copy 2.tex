\documentclass{beamer}

\usepackage{beamerthemesplit}
\usepackage{wasysym}
\usepackage{amsmath, mathtools}
\usetheme{Berlin}

\title{Damage Theory}
\author{Jessica Cervi}
\date{\today}


\begin{document}
\begin{frame}
\maketitle
\end{frame}
 
\section{Damage Theory}
\subsection{Background}
\begin{frame}
\frametitle{Classical DamageTheory}
Classical theory (widely accepted, Saparto and Dewey, 1984) is based on the assumption that damage  is measured by 
\begin{equation*}
TD(t) = \int_0^tr^{(43-T(t))}dt \qquad r = \left\{\begin{array}{cc}0.25 & T\le43^oC\\0.50&T>43^oC\end{array}\right.
\end{equation*}

This formula 
\begin{itemize}
\item is entirely phenomenological/heuristic (has no mechanistic basis);
\item Damage threshold varies widely among different tissues;
\item No explanation of the significance/origin of the threshold at 43$^o$C.
\end{itemize}



\end{frame}

\begin{frame}
\frametitle{Another Idea}
In paper by  Zhou, Chen, and  Zhang, 2007, it was suggested that damage is the result of irreversible, protein denaturization, governed by the chemical reaction
\begin{equation*}
P\rightarrow D,
\end{equation*}
($P =$ folded protein, $D=$ denatured protein) at an Arrhenius reaction rate
\begin{equation*}
r(T) = A\exp(-\frac{\Delta G}{RT})
\end{equation*}
where $\Delta G$ is activation energy, $R$ is universal gas constant.

This leads to damage fraction
\begin{equation*}
\Omega(t) = \log(\frac{P_0}{P(t)}) = \int_0^tA\exp(-\frac{\Delta G}{RT(t)})dt
\end{equation*}
($P_0=$ initial folded protein).
 
\end{frame}
 

\begin{frame}
\frametitle{Comments}
\begin{itemize}
\item While this formula was used for damage from a laser, we believe it is also applicable to damage from ultrasound (HIFU);
\item The parameters $A$, $\Delta G$ and $\Omega_\theta$ (damage threshold) can be chosen to match different tissue types.
\item $\Omega(t)$ can be readily computed using the Pennes model (described above).
\end{itemize}
 
\end{frame}
\end{document}
 
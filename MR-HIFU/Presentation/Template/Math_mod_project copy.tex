\documentclass{beamer}

\usepackage{beamerthemesplit}
\usepackage{wasysym}
\usepackage{amsmath, mathtools}
\usetheme{Berlin}

\title{Advection and diffusion in a porous medium}
\author{Jessica Cervi}
\date{\today}


\begin{document}
\begin{frame}
\maketitle
\end{frame}
\begin{frame}
 \tableofcontents
\end{frame}

\section{The model}
\subsection{Background}
\begin{frame}
\frametitle{Porous medium model}
\begin{itemize}
\item What is a \alert{Porous medium?}
\end{itemize}
\vspace{1cm}
 \begin{columns}
      \begin{column}{0.5\textwidth}
 Tissues can be treated as a porous medium as they are  composed of dispersed cells separated by connective voids which allow for flow of nutrients.
      \end{column}
      \begin{column}{0.4\textwidth}
         \includegraphics[scale=0.4]{porous_med}
      \end{column}
   \end{columns}


\end{frame}

\begin{frame}
\frametitle{Mass diffusion and advection in brain tissues}
\begin{itemize}
\item \alert{Diffusion} is an essential mechanism for delivering glucose and oxygen from the vascular system to brain cells.
\item \alert{Advection} is a transport mechanism of a substance or conserved property by a fluid due to the fluid's bulk motion.
\end{itemize}
\end{frame}



\subsection{Geometry}
\begin{frame}
\frametitle{The domain}
We model a simplified situation in which the fluid is moving through a 2D thin fracture that could be seen as an oversimplified porous medium,\\The domain is the form:
\begin{align*}
\Omega=\{\mathbf{x}=(x,y,)\in \mathbb{R}^2,0<x<L, 0<y<H_0\}
\end{align*}
where $H_0$ is the height of the channel and $L$ is the characteristic lenght of the domain.
The following assumption holds:
\begin{align*}
H_0\ll L \to \epsilon=\frac{H_0}{L}\ll1
\end{align*}
\end{frame}
\begin{frame}
\frametitle{Mass diffusion and advection in brain tissues}
\begin{figure}
\includegraphics[width=10cm, heigh=5cm]{porous1}
\end{figure}
\end{frame}


\begin{frame}
\frametitle{Boundary and initial condition}
Boundary condition:
\begin{itemize}
\item No-slip boundary condition applies on the bed surface
\begin{align*}
u_y=0 \qquad \text{on   } y=0, y=H_0 
 \end{align*}
\end{itemize}
Concentration initial condition is:
\begin{align*}
c=c_0\qquad \text{  at    }t=0
\end{align*}
where $c_0$ is a smooth function, indicating that in a typical cell of the porous medium the initial concentration is uniform.
\end{frame}

\section{Governing equations}
\subsection{Continuity equation}
\begin{frame}
\frametitle{Advection-diffusion equation}
\framesubtitle{Outline}
The transport of oxygen can be described by the advection-diffusion equation. It be derived from the \alert{continuity equation}.
\begin{align*}
\frac{\delta c}{\delta t}+\nabla \cdot \mathbf{J}=0
\end{align*}
where $c$ is the concentration of the species and $\mathbf{J}$ is the flux. \\
Sources of flux:
\begin{itemize}
\item diffusive flux
\item advective flux
\end{itemize}
\end{frame}

\begin{frame}
\frametitle{The Fick's law}
\framesubtitle{Diffusive flux}
Fick's law describes concentration $c $ of a compound in a material
\begin{align*}
\frac{\delta}{\delta t}\left(\iiint_V c \delta V\right)=-\oiint_S D \nabla c \delta S
\end{align*}
 Using the divergence theorem and recalling that $V$ is an arbitrary volume we can rewrite the above equation as:
\begin{align*}
\frac{\delta c}{\delta t} =-D \nabla \cdot \left( \nabla C_f\right) =-D\nabla^2 C_f
\end{align*}
where $c$ is the concentration of the nutrients .
\end{frame}

\begin{frame}
\frametitle{Advective flux}
The advection is the transport mechanism of a species  by a fluid due to the fluid's bulk motion. \begin{align*}
\frac{\delta c}{\delta t}=\nabla \cdot ( \mathbf{u}c)
\end{align*}
where $\mathbf{u}=(u_x, u_y)=(u_x,0)$ is the blood flow term (i.e the velocity vector field). 
\end{frame}


\begin{frame}
\frametitle{The uptake term}
The uptake term  represents the consumption of the species by the surrounding tissue. This can be modeled by the following equation:
\begin{align*}
F(c)=F_0c
\end{align*}
where $F_0$ is the surface reactivity coefficient.

\end{frame}

\begin{frame}
\frametitle{The advection-diffusion equation}
\framesubtitle{Final form}
Conservation of $c$ reads:
\begin{align*}
\frac{\delta c}{\delta t}+\nabla \cdot \left(\left(\mathbf{u} c\right)-D^*\nabla c\right)=F(c)
\end{align*}
where $D^*$ is the {\em effective diffusion coefficient} given by:
\begin{align*}
D^*=\frac{\phi D}{\tau^2}
\end{align*}
where $\phi$ is the porosity and $\tau$ is the tortuosity of the capillaries
\end{frame}

\subsection{Darcy and Brinkman models}
\begin{frame}
\frametitle{The blood flow model}
The blood flow term could be derived starting from the  Navier-Stokes equations for incompressible fluids:
\begin{align*}
\rho \frac{\delta \mathbf{u}}{\delta t}+(\mathbf{u}\cdot\nabla)\mathbf{u}\rho&=-\nabla P+\nu \nabla^2 \mathbf{u}+F\\
\nabla \cdot \mathbf{u}&=0
\end{align*}
\end{frame}

\begin{frame}
\frametitle{The blood flow model}
\framesubtitle{Asymptotic expansion}
The velocity $\mathbf{u}$ and the pressure $P$  can be written  as  asymptotic expansions in $\epsilon$:
\begin{align*}
&\mathbf{u}^{\epsilon}=\epsilon^2\mathbf{u}^0+\epsilon^3\mathbf{u}+O(\epsilon^4)\\
&P^{\epsilon}=P^0+\epsilon P+ \epsilon^{2}P^2+O(\epsilon^3)
\end{align*}
\end{frame}

\begin{frame}
\frametitle{The blood flow model}
\framesubtitle{Asymptotic expansion}
For a slow steady flow we ca neglect the left-hand term side in the momentum equation. Substituting the expansions and collecting equal power of $\epsilon$ yield:
\begin{align*}
&\Big[-\nu\frac{\delta^2 \mathbf{u}^0}{\delta y^2}+\nabla_{x'}p^0+\frac{\delta P^1}{\delta y}\mathbf{k}\Big]+
\epsilon\Big[-\nu\frac{\delta^2 \mathbf{u}^1}{\delta y^2}+\nabla_{x'}p^1+\frac{\delta P^2}{\delta y}\mathbf{k}\Big]+\\
&+\epsilon^2 \Big[-\nu\frac{\delta^2 \mathbf{u}^2}{\delta y^2}-\nu\nabla_{x'}\mathbf{u^0}+\nabla_{x'}p^2+\frac{\delta P^3}{\delta y}\mathbf{k}\Big]+\cdots=F
\end{align*}
\end{frame}

\begin{frame}
\frametitle{The blood flow model}
\framesubtitle{Leading order solution}
Keeping only the first term we arrive at:
\begin{align*}
-\nu\frac{\delta^2 \mathbf{u}^0}{\delta y^2}+\nabla_{x'}p^0+\frac{\delta P^1}{\delta y}\mathbf{k}=F
\end{align*}
Regarding that $\mathbf{u^0}=0$ for $y=0$ the solution is in the form:
\begin{align*}
\mathbf{u}^0=\frac{1}{2\nu}y(h(x')-y)\mathbf{w}(x')
\end{align*}
\end{frame}

\begin{frame}
\frametitle{The blood flow model}
\framesubtitle{Leading order solution}
The standard Darcy's law:
\begin{align*}
\mathbf{w}+\nabla_{x'}P^0=F
\end{align*}

Darcy model \alert{ignores the boundary effects} on the flow. 
\end{frame}

\begin{frame}
\frametitle{The blood flow model}
\framesubtitle{Lower order terms}
The equation for the first and the second order are:
\begin{align*}
O(\epsilon)\to & -\nu\frac{\delta^2 \mathbf{u}^1}{\delta y^2}+\nabla_{x'}p^1+\frac{\delta P^2}{\delta y}\mathbf{k}=0\\
& \nabla \cdot \mathbf{u^0}+\frac{\delta u_2^1}{\delta y}=0
\end{align*}
\begin{align*}
O(\epsilon^2) \to& -\nu\frac{\delta^2 \mathbf{u}^2}{\delta y^2}-\nu\nabla_{x'}\mathbf{u^0}+\nabla_{x'}p^2+\frac{\delta P^3}{\delta y}\mathbf{k}=0\\
&\nabla \cdot \mathbf{u^1}+\frac{\delta u_2^2}{\delta y}=0
\end{align*}
\end{frame}
\begin{frame}
\frametitle{The blood flow model}
\framesubtitle{Lower order terms}
Order $\epsilon$ solution:
\begin{align*}
O(\epsilon)\to \begin{cases}
&u_3^1 = -\int_0^y(div_{x'}\mathbf{u}^0(x',z)\delta z)\\
&\mathbf{u}^1  = u_3^1\mathbf{k}\\
&p^2 = \nu\frac{\delta u_3^1}{\delta y}
\end{cases}
\end{align*}
\end{frame}

\begin{frame}
\frametitle{The blood flow model}
\framesubtitle{Lower order terms}
Applying the previous solution to the order $\epsilon^2$ system leads to:
\begin{align*}
O(\epsilon^2)\to \begin{cases}
&\mu B^{\epsilon}\mathbf{u}^{\epsilon}+\nabla_{x'}p^0-\nu \nabla^2 \mathbf{u}=F\\
&div_{x'}\mathbf{v^{\epsilon}}=0 \text{ in the domain  }O\\
&\mathbf{u}^{\epsilon}=0 \text{ on } \delta O
\end{cases}
\end{align*}
Which is the Darcy-Brinkman law for viscous fluid \alert{with no-slip B.C}.
\end{frame}

\begin{frame}
\frametitle{The blood flow model}
\framesubtitle{Comparison between the models}
\flushleft
\includegraphics[scale=0.45]{table}
\end{frame}

\section{Conclusion and future work}
\begin{frame}
\frametitle{Remarks}
\begin{itemize}
\item Diffusion and advection are both important to model the transport of nutrient through tissues.\\
\item The uptake term takes into account the loss of nutrients and the chemical reactions.\\
\item The Brinkmann model is more accurate and efficient than Darcy model.\\
\end{itemize}
\end{frame}

\begin{frame}
\frametitle{Future work}
\begin{itemize}
\item Model the uptake term considering the chemical effects
\item Combine the diffusion-advection equation with the Brinkman model to find an expression for the concentration\\
\item 3-D model and simulation
\end{itemize}
\end{frame}
\begin{frame}
\centerline{THANKS!}

\end{frame}
\end{document}

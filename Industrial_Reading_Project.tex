
%----------------------------------------------------------------------------------------
%	PACKAGES AND OTHER DOCUMENT CONFIGURATIONS
%----------------------------------------------------------------------------------------

\documentclass[11pt]{article} % Default font size is 12pt, it can be changed here

\usepackage{geometry} % Required to change the page size to A4
\geometry{a4paper, margin=2cm} % Set the page size to be A4 as opposed to the default US Letter

\usepackage{graphicx} % Required for including pictures

\usepackage{float} % Allows putting an [H] in \begin{figure} to specify the exact location of the figure

\usepackage{cite}
\usepackage{amsmath}
\usepackage{gensymb}
\usepackage{url}

\begin{document}
	\begin{titlepage} % Suppresses displaying the page number on the title page and the subsequent page counts as page 1
		
		\raggedleft % Right align the title page
		
		\rule{1pt}{\textheight} % Vertical line
		\hspace{0.05\textwidth} % Whitespace between the vertical line and title page text
		\parbox[b]{0.75\textwidth}{ % Paragraph box for holding the title page text, adjust the width to move the title page left or right on the page
			
			{\Huge\bfseries Reading Project: \\[0.5\baselineskip] Industrial Mathematics}\\[2\baselineskip] % Title
			{\large\textit{Margaret Duff}}\\[3\baselineskip] % Subtitle or further description
			{\large{Semester 1 2018}}\\[3\baselineskip] % Subtitle or further description
			{\Large\textsc{Supervised by: Professor Chris Budd}}\\[4\baselineskip] % Author name, lower case for consistent small caps
			{ABSTRACT}\\[1\baselineskip]
			{This report forms the conclusion of a reading project completed at the University of Bath supervised by Professor Chris Budd. 
				The first part of the report focuses on the International State of Industrial and Applied Mathematics, Mechanisms, Philosophy and Effectiveness and includes 4 cases showing just some of the breadth of applicable mathematics research carried out. The second part of the report looks at a specific Industrial Mathematics problem, that of modelling temperature variations due to High Intensity Focused Ultrasound on structures including human muscle and bone. A background to the problem is given, along with examples of previous work before some current work is explained.  }\\[4\baselineskip]
			
			
			
		}
		
	\end{titlepage}

\pagebreak
	\tableofcontents 
	\pagebreak
	\section{Introduction}
	
	\section{Review of the International State of Industrial and Applied Mathematics, Mechanisms, Philosophy and Effectiveness}
	
	In this section we will look at the International State of Industrial and Applied Mathematics, Mechanisms, Philosophy and Applied Mathematics. This report will mainly be  focused on the UK but we will look elsewhere in the world for examples and comparisons. 
	
	We start by attempting to define Industrial Mathematics before questioning why it is important  and how it is done. Finally I will make some judgement of successes of Industrial Mathematics  and some of the challenges facing its future. Case studies will be discussed and referenced throughout to add provide concrete, real world references. 
	
	\subsection{Definitions} 
	
	Defining the language and descriptors of  Industrial Maths is not a simple task and indeed a variety of distinguished writers choose a range of definitions. 
	
	Even defining what it means to do Mathematical Research is fraught with complications. Even counting all those employed by universities and higher education institutes misses a large number of very talented mathematicians working elsewhere. 	Deloitte in their report 'Measuring the Economic Benefits of Mathematical Science Research in the UK' \cite{deloitteuk} count 'Mathematical Science Occupations' as those 'which either entail mathematical science research, or used mathematical science research-derived tools and techniques' a broad definition which includes individuals which may use mathematically derived techniques but have no understanding of the underlying mathematics, For example they includes all hospital and healthcare managers, social science researchers and public service administrative professionals, all very valuable jobs but not ones that I would put in the category of 'Mathematical Science Occupations'. 
	
	
	In a similar report Doloitte produced for the Dutch economy \ref{deloittedutch} they narrow this definition slightly by considering 'only people in jobs requiring a higher education' to be included.  A better but still quite broad definition and probably still includes healthcare managers and social science researchers!
	
	
	One might suggest considering only those people who currently use or research 'modern' mathematical science research tools, to avoid including all those people that use an excel spreadsheet to add up large numbers because "addition" is a mathematical technique. But then the definition of 'modern' is fraught: number theory developed during the end 19th and beginning of the 20th century underpins the very modern phenomenon of internet shopping. The radon transform, developed at the beginning of the 20th century, is behind most modern medical imaging. The motion of a rocket from the surface of the Earth to a landing on the Moon can be explained and described by physical principals discovered over 300 years ago by Sir Isaac Newton.
	
	
	For the purpose of this report, I  will consider Mathematics to be that which is of interest to academic mathematicians; it focuses  on the underlying structure and pattern of a problem and  looks for generalisations and derives exact or approximate solutions backed up by logical proof. The important clarification is that the application of well known techniques does not count. (Finally discounting those Social Scientists using statistical software!)
	
	
 Industry  I define it be any non-mathematical institution including: governments, businesses, manufacturers, other academic departments, hospitals, schools, charities etc. 


		A wealth of words are used as synonyms or colloquialisms for industrial mathematics: 'applicable', 'interdisciplinary, 'applied', 'knowledge transfer' or 'mathematics communication'. The Bond Report \cite{Bond} describes '\textit{Impactful Mathematics} as any mathematical method that has practical application and generates societal and/or economic value'. 
		
		For this report I will use the definition John Stockie describes in his essay 'Mathematics for Industry A personal Perspective' \cite{Stockie2015} that Industrial Mathematics includes: 
		
	\begin{itemize}
	\item mathematics that is done by non-academic mathematicians who work as employee of a company
	\item mathematics that is done by academic mathematicians within research institutions for a company, or in collaboration with a company
	\item mathematics that is inspired by industry and arise from an industrial setting. 
	\end{itemize}
	
	
	\subsection{Significance}
	
Although most people don't grow up to become mathematicians, they do grow up to rely on mathematics: from weather forecasts to smart phones; from the logisitcs of ensuring fresh food in supermarkets to online shopping or from the mechanics of a car engine to  the ability to understand the mysteries of the universe. Mathematicians like to tell you that "Maths is Everywhere" and indeed it is! 

However, a more quantitative argument is required to convince the most sceptical of readers. The Deloitte Report on the Economic benefits of mathematical research on the Uk economy \cite{deloitteuk} although perhaps over estimating, as discussed above, suggests that a total of 9.8million people are in employment attribute to mathematical science research, of which: 2.8 million can be directly attributed to mathematical science research, 2.9 million can be attributed indirectly through 'supply chain' impacts e.g. support staff, cleaners, technicians etc. and finally 4.1 million people have employment induced  through spending by households linked to mathematical science research. They also concluded that in 2010, mathematical science research in the UK generated direct gross-value added (GVA, value of the output less the value of the immediate consumption ) of approximately £208 billion, or around 16\% of total UK GVA.  Productivity, GVA per worker, in Mathematical Science occupations in 2010 was calculated at £74,000 per worker which can be compared to the UK productivity average in 2010 which was estimated to be £36,000 and the mean wage for mathematical science research jobs over £44,000. 
 
 These results evidence the significance of Mathematical Science Research on the UK economy and therefore how important that it continues to grow and remain competitive in the future. 
 
 
	
	SOMETHING ON THE Dutch Deloitte Report??
	
	 
	\subsection{Current State }
	
	\subsubsection{ Mechanisms in the UK  }
	
	The Bond report \cite{Bond} found from its call for evidence a list of the most commonly cited examples of industrial mathematics mechanisms. This included: 
	\begin{itemize}
		\item \textit{Industrial Cooperative Awards in Science and Technology (CASE) PhD studentships }- A company allocated an award defines a research project and picks an academic partner. Once the arrangements for the project have been agreed between the company and research organisation, they can recruit a student. Students receive funding for a full EPSRC studentship for 4 years topped up by the company. A placement of at least 3 months at the company is required and projects should be in the area of engineering and the physical sciences. Students benefit from access training, facilities and expertise not available in an academic setting alone and academic partner institutions benefit from novel research collaborations and  development/ strengthening  of partnerships. Companies that were awarded EPSRC ICASE allocations for 2019/20 include the Defence Science and Technology Laboratory, Rolls-Royce PLc and the National Physical Laboratory. https://epsrc.ukri.org/skills/students/coll/icase/intro/
		
		\item \textit{Knowledge Transfer Network Ltd activities }- funded by innovate UK provides innovation network for other funders in line with its mission to drive UK growth. Examples of their programmes include: 
		\begin{itemize}
			\item \textit{ The 4manufacturing initiative } provides strategies for businesses to identify their barriers, challenges and next steps on the path to integrating modern technologies and systems to use digital data within both the business and the supply chain
			\item \textit{Knowledge transfer partnerships} (see below )
			\item \textit{Special Interest Groups} projects that focus our activity and accelerate innovation in cross-disciplinary topics of strategic importance. Groups funded included: Sustainable Aviation Fuel, Synthetic Biology, and Uncertainty Quantification and Management.
			\item \textit{The Access to Funding and Finance team} can help businesses understand and raise funding and finance – lending, grants, or equity based, as well helping prepare companies for investment.
		\end{itemize}
		\item \textit{Mathematics Study Groups} - Provide a forum for industrial scientists to work alongside academic mathematicians  on problems of direct industrial relevance. Initiated in Oxford in 1968 the format has been copied around the world and in even extending into other areas. The structure is as followed: 
		\begin{itemize}
			\item Approximately 80 mathematicians from a wide range of backgrounds
			\item On the first day the industrial representatives outline their project and aims
			\item The following few days are devoted to solving the problems. The industrial representatives are available to answer questions and guide the projects.
			\item On the last day, any progress made is presented
			\item Reports in the problems are produced after the meeting  
		\end{itemize}
		There are many benefits to industry including: 
		\begin{itemize}
			\item Access to the expertise of leading applied mathematicians to their problem
			\item New perspectives and fresh ideas on their problems
			\item Establishing links with research mathematicians
			\item An opportunity to formulate and reflect on problems  of long-term significance
		\end{itemize}	
		The case study in section \ref{homeless} is taken from the 29th ESGI held in Oxfords in March 1996. There were also many that I found extremely interesting but did not have time to investigate further such as  "Optimization of ATMs filling-in with cash" \cite{Broda} and "Improving defrosting procedure for a frozen dough" \cite{Luci}. All the cases studies demonstrate the broad range of problems considered at these Study Groups. 
		\item \textit{The Turing Gateway to Mathematics (TGM)} is the impact initiative of the Isaac Newton Institute (INI) based at the University of Cambridge. The TGM reaches out to and engages with the users of maths: industry, business, public sector and other scientific disciplines. Maintaining contacts across the world they can facilitate interactions and activities such as programmes of work, research and training events, as well as bespoke projects
		\item  \textit{A Knowledge Transfer Partnership} helps facilitate a 3 way partnership between a business or not for profit organisation, a research organisation or university and a graduate (known as the Associate). Designed for businesses who have an idea for a new project, capability or significant change but  do not have all the in-house expertise. Businesses benefit from new expertise and innovation. Research organisations benefit financially and  the possibility of publishing, identifying new research themes and apply their knowledge and expertise to real-world problems. Associates gain experience, often job opportunities and dedicated coaching, mentoring and personal development through working with the KTN.
		\item \textit{Heilbronn Institute}
		 
		
	
	\end{itemize}

A couple of examples not discussed in the Bond report include:
\begin{itemize}
	\item \textit{Catapult Centres}: perhaps not included because it is not aimed primarily at mathematical research but I believe that Catapult Centres are an interesting model, for academic and industry collaboration. They are a series of physical centres where  businesses, scientists and engineers can work side by side on late-stage research and development – transforming high potential ideas into new products and services to generate economic growth.  The aims is to:  Catapults exist to:reduce the risk of innovation; 	accelerate the pace of business development;  create sustainable jobs and growth and to	develop the UK’s skills and knowledge base and its global competitiveness. The centres gain their funds from a mix of competitively earned commercial funding and core Innovate UK investment and currently 11 centres exist in areas including: energy systems, transport systems and satellite applications. 
	\item \textit{The Smith Institute} for Industrial Mathematics and Systems Engineering specialises in solving complex tactical and strategic problems for businesses, governments and organisations \cite{Smith}. They act as a mathematical consultancy service and have worked with a variety of busineeses including looking at:  predicting the weathers impact on demand of Coca-Cola; adaptive bidding strategies with Vodafone; and predicting the energy output from embedded generation with the National Grid. 
\end{itemize}
Another avenue not discussed in the Bond Report but one that I think is hugely important is that of Undergraduate placements, internships and sandwich courses. Organised well this could be a way of companies benefiting from mathematical expertise at a comparatively cheap rate and of students gaining the first taste of industrial mathematics. A more competitive graduate jobs market also means that these placements are highly demanded by students and becoming increasingly common. 
\begin{itemize}
	\item \textit{Undergraduate placements and sandwich courses}- These can be anything from a couple of weeks to a full year in industry, organised by the individual student or with the support of their university and their industrial contacts, undergraduate placements match an eager group of undergraduate mathematicians with industry. Students develop transferable skills and gain experience doing innovative mathematical research in an external setting, while hosts engage a talented mathematician to tackle research questions they face.
\end{itemize}




NEED SOMEWHERE TO TALK ABOUT THIS!!!!!!!!!!!
\subsubsection{Incentives and Enablers in the UK} 
A major incentive for UK Higher Education Institutions to engage in industrial mathematics is the 20\% impact component of the Research Excellence Framework(REF). The goal of the REF is to assess the quality of research in UK higher education institutions. For each submission, three distinct elements are assessed: the quality of outputs (e.g. publications, performances, and exhibitions), their impact beyond academia, and the environment that supports research. The REF outcomes are used to inform the allocation of around £2 billion per year of public funding for universities’ research.  This gives an incentive for institutions to aim for higher results in the REF including higher scores in the Impact component. For the 2014 REF 6975 impact case studies were submitted, 95\% of which were in statistics and operational research and about 5\% in pure mathematics but many encompassing difference aspects of the mathematical sciences. Shining examples of case studies were collated in the book " Uk success stories in Industrial Mathematics", \cite{Aston2016},  which we reference repeatedly and from which the case study in section\ref{Africa} is taken from. Impact has been considered a beneficial addition to the REF and will be worth 25\% in the REF2021 assessment \cite{REF2017}. The REF looks at all types of research, not just Mathematics (SOMETHING TO SAY??)

The introduction of the "Impact" Category to the REF in 2014 is reflective of the Research Councils and Higher Education Funding Councils' increased emphasis on research impacts. Operating across the whole of the UK with a combined budget of more than £6 billion, UK Research and Innovation brings together the seven Research Councils, Innovate UK and a new organisation, Research England. They encourage all grant applications to consider  what can be done to ensure that research makes a difference both in an academic sense, improving understanding, theory methods and applications but also in a broader way looking at the economic and societal impacts. These embrace all the extremely diverse ways in which research-related knowledge and skills benefit individuals, organisations and nations \cite{UKRI}. 

An important source of funding, discussed by Professor Dame Ann Dowling, in her report \ref{Dowling2015}, is Impact Acceleration Accounts. IAAs are one of the funding mechanisms for supporting knowledge exchange, innovation and impact at UK Research and Innovation and are available from several Research councils with example uses including: providing proof of concept funding, employing staff to focis on creating research impact and facilitating industrial access to research or to enable university staff to work in businesses or on secondment. Professor Dowling writes that they are "particularly valued for the speed with which the funding can be mobilised and deployed".  EPSRC currently funds 33 IAAs across the UK, with a total investment since 2012 of more than £150 million \cite{IAA}.
		
Things are definitely moving in the right direction....MORE HERE
	\subsubsection{Elsewhere in the world }	
	
	We also look for examples elsewhere in the world. The Bond report \cite{Bond}, makes a comparison with Germany.   In addition to universities, in Germany there are two further groups that play an important role in applied research:
	\begin{itemize}
		\item \textit{Universities of Applied Sciences} focus on teaching professional skills and importantly appointment to a professorship is only possible with those with a track record in industry. Mathematics groups tend to be focused on applied mathematics, statistics and operational research and can be embedded in engineering departments. 
		\item \textit{Fraunhofer institutes} focus on a particular field of applied research in areas including information and communication technology, life sciences, lights and surfaces and defence and security. There are 72 institutes across Germany employing some 25,000 people the majority of which are qualified scientists and engineers. Around 70 percent of Fraunhofer’s contract research revenue is derived from contracts with industry and publicly funded research projects, while about	30 percent of Fraunhofer’s budget is accounted for by
		base funding provided by the German Federal Ministry of	Education and Research (BMBF) and the state governments. They also work to set up spin-offs, setting up 25 start ups by 2017 \cite{Thum2017}. They also undertake international activities, for example the Kaiserslautern Graunhofer Institute in Industrial Mathematics has an associated centre in Gothenburg, Sweden and the Frauhofer Institure for Applied Photonics is based at the University of Strathclyde. 
	\end{itemize}

	Another noteworthy institution is the Steinbeis Foundation: an institution promoting the transfer of technology and knowledge between universities and the industry. It supports the formation and operation of spin-put companies, and from the end of 2017 had 1072 enterprises in their network in 2017 \cite{Steinbeis}.
	
	With a similar sized population to the UK, Germany filed 18,305 international patent applications (PCT) in 2016 compared to 5,501 in the UK as rpeorted by the World Intellectural Property Organisation \cite{WIPO}. This places Germany 4th in the ranking of the 10 countries who filed the most international patent applications in 2016, suggesting the system as a whole is effective. 
		
	Looking elsewhere in the word, we look to Canada and two mechanisms for Industrial Mathematics: 
	

	\begin{itemize}
		\item \textit{Fields Institute and it's Commercial and Industrial Mathematics (CIM) Program} \cite{FieldsCIM}  has a focus to cooperate with business, enabling technology transfer between mathematical scientists and the information society.The program supports several seminar series, occasional workshops, courses and talks, and the Fields Industrial Problem-Solving Workshop (IPSW), which takes place at Fields in even-numbered years. The IPSWs in Canada began 20 years ago in Western Canada with the PIMS workshops, and are loosely modelled on the “study groups” pioneered by the University of Oxford, discussed previously. The 1998 IPSW is the source of the case study described in Section \ref{landmines}.
		 In addition, affiliated with  the Fields Institute Mathematics-in-Industry Case Studies Journal was launched, although (ASK CHRIS WHAT IS HAPPENING) 
		\item \textit{Centre for Financial Industries}: many former CIM activities in financial mathematics are now organized through the Centre for Financial Industries \cite{FieldsCFI}. Activities include an ongoing world-renowned monthly seminar series, a highly successful six month thematic program, and a variety of workshops, conferences, short courses and industrial-academic events. 
		\item \textit{Fields Centre for Quantitative Analysis and Modelling (Fields-CQAM)} is another creation by the Fields Institute \cite{CQAM}. It brings together a network od academic-industry collaborative laboratories, their international collaborators and  student trainees working   with industrial partners like Sanofi-Pasteur, RANK Software, and Nuralogix to solve commercial problems using applied mathematical research. It aims to efficiently commercialise research for the benefit of the economy while training the next generation 
	\end{itemize}
	

	
With a population of about 1/2 that od the UK, Canada had 2,332 PCT applications in 2016 \cite{WIPO} putting it on par with the UK, but of course this is much more general than just mathematical sciences. MORE HERE>>>>
	\subsection{Successes}
	 
	
	\subsection{Failures} (GET RID OF THIS SECTION???)
	Following a section on successes it makes sense to look at failures, however these are much harder to find, unlikely to be published or advertised. 
	
	Mathematics-in-Industry Case Studues Journal 
	
	\subsection{Challenges} 
	In this section we consider a range of challenges to successful Industrial Mathematics. 
	
	\subsubsection{Identifying Suitable Problems and Creating links} 
	
	There is always going to be difficult identifying suitable problems, which suit the skill set of the individuals involved, are managable in the time frame and have a chance of being successful. 
	
	Especially, in the case of undergraduate placements and internships is is extremely difficult for companies and individuals, both of which may have little experience of doing so, of formulating a problem that is achievable but suitably stretching for the individual. Everyone has heard stories of making tea and doing the photocopying during internships but potentially a less visible problem is the "code monkey" issue. The Cambridge Mathematical Placements scheme asks all students returning from placements whether the placement would be more suitable for someone with less mathematical but more coding skills. Students are gaining valuable skills and companies are benefiting from bright enthusiastic individuals but they aren't doing mathematics. An interesting thought. 
	
	
	
	When initiating colaborations it can be hard for individual academics to know where to start, there may be  no recognised route and they can be  unsure of their own abilities 
	
	The Bond report quoted that 'just 26\% of SMES and large business respondents cited via a University knowledge/technology transfer office" as their mode of initiation. Initial contacts can be based on a who-you-know  method; indeed John Stockie describes how is first bit of industrial mathematics developed out of a  "neighbourhood social event" \cite{Stockie2015}. Large alumni networks can be incredibly useful in initiating contacts and this is highlighted in the Bond report \cite{Bond}  which recommends "mechanisms to generate systematic and long-term relationship building and engagement with alumni should be created."
	
	It also recommends that more be done  and "a more systematic and coordinated approach needs to be adopted to make new and maintain existing KE [ knowledge exchange] contacts and to track the outcomes and impacts of KE activities". A more systematic approach may be better able to allocated resources to be most effective, tackling the most important problems in which significant progress is most achievable. To do this requires some sort of oversight and indeed the Bond report recommends: " An Academy for the Mathematical Sciences should be established in order to facilitate links between academia, government and industry". An excellent idea, but in practice requires funding, and is only really feasible in the long term as relationships and trust take a long time to build up. 
	
	From a business perspective the costs of engaging with academia in some of the routes discussed above may be unmanageable. For example, ESGI generally charge industrial partners  £7500 (as of 2017) in the UK or 5000  euros (as of 2017) in Ireland \cite{ESGIhandbook}. Fees for mathematical consultancy services such as the Smith Insitute and Bath University's Instiute for Mathematical Innovation will be substantially higher and although Bath Universities Integrative Think Tanks have no upfront costs, it is expected that companies will be able to fund PhD places if such projects are developed out of the Think Tank. 
	
	There are sources of funding available especially for small and medium sized businesses (SMEs) \cite{IMI}. For example an SME can claim up to 230\% of its research and development costs \cite{Gov}  and there are grant schemes that offer funding to SMEs to engage in research and development projects in the strategically important areas of science, engineering and technology. Innovation Vouchers are a UK government scheme that	provides up to £5,000 to help small or medium-sized
	businesses access external expertise. Many institutions are also willing to reduce costs if an interesting problem is posed by a SME. More needs to be done to make all parties aware of these avenues and the Bond review \cite{Bond} highlights that although much collaboration between the mathematical sciences can be considered an "advance in science and technology" and thus eligible for tax relief it would be better if "the mathematical sciences should be encompassed in the HMRC definition of science and technology".
	
	
	\subsubsection{Building working relationships}
	
	There are different drivers, incentives and cultures in industry and in academia and this can cause conflict or slow down collaboration. Examples include: 
	\begin{itemize}
		\item \textit{Timescales} - In general academia works on much longer timescales than industry would like. For example a CASE PhD project is going to be at minimum 3 years, if not more. A company looking for a quick return on investment or technology for the next season are looking at much shorter time frames. Resourcing of projects at short notice is also typically difficult for university groups. In response to enquiries for the Bond report \cite{Bond}, just 58\% of business respondents said they were able to access support quickly enough. 
		\item \textit{Technical Language} -  more of a problem when dealing with large companies where you sometimes find an internal language and series of acronyms, a type of slang. This can feel excluding to individuals coming in but also provide a barrier to understanding. A simple example is a recent visit to the University of Bath by Willis Towers Watson, an insurance brokers, to discuss some of their current challenges. Some of the mathematicians were confused by WTW's use of the word "independent" when describing an index to mean that the index was not controlled or influenced by either the insurer or the insured party, wheras the mathematicians were thinking about the use in probability theory and couldn't understand how a volcanic index was "independent" from a volcanic event!
		\item \textit{Criteria for Success} - A common joke, making fun at mathematicians describes their response to discovering a fire at night: they looks at the fire, looks at the fire extinguisher, and thinks for a minute, says "Ah! A solution exists!" and goes back to bed. A solution by someone in industry might be to estimate how much water is needed to put the fire out and then mulitply this amount by 10 just to make sure. These are both gross stereotypes but do reflect a differing culture between mathematics and industry, that can effect relationships.		
	\end{itemize}

	
	\subsubsection{Intellectual Property and Conflicts of Interest} 
	Universities have  become more aware of the importance of intellectual property and have significantly professionalised their knowledge exchange activities. However, there is a tension between the desire to earn short-term income from their IP and the need to deliver wider public benefit, and potentially greater long-term return on investment from this IP. Some businesses also rely on IP to remain profitable, so with any work undertaken as a collaboration with industry agreements need to be made on OP and confidentiality. The Bond Report \cite{Bond} suggests that that in 48\% of interactions with large businesses reaching agreement on IP was reported as a barrier.   Infelxibility of universities on contractual issues can put a complete stop to a project. 
	
	 The Dowling review \cite{DOWLING2015} suggests that universities should prioritize industrial callobaration over short term revenue generation but this is easier said than done. 
	
	Another factor that can discourage academics from pursuing collaborations is discussed by Professor Dowling in her report \cite{DOWLING2015} is concern that accepting industrial funding for research may make the researcher vulnerable to accusations of conflicts of interest, especially if there is media interest in the story. A recent example is that of  criticisms of public health experts in receipt of funding from the sugar industry. Something that individual universities should be able to provide guidance on. 
	\subsubsection{Academic Career Paths} 
	Discussed heavily in the Bond Report: demand for mathematical experience is ever increasing especially in areas such as AI, machine learning, genomics, data science and finance. Skilled people can demand high salaries so academic departments need to change: to attract students they need to train individuals is these mathematical areas as well as in less traditional business and leadership skills and to attract academics they need to provide flexible, interesting  and remunerative career paths. 
	
	Free movement of skilled people between academia and industry is important to make and maintain links, build skills and create a supportive environment (LOOSE THAT THIRD ONE). 
	
	However time spent in industry is often seen as detrimental to an academic career progression, with its strong emphasis on sheer quantities of research publications and citations. 
	
	Academics often have many calls on their time and it is essential that adequate time, recognition and resources is available to enable and incentivise interaction with industry. Work done with industry must be recognised and rewarded by institutions. 
	
	Industrial mathematics is often time intensive with perhaps questionable clear rewards. Contacts take time to build and maintain and the time taken to do this is not generally recognised. Expected to do this on top of everything else. Funding for proof-of-concept/ sky blue thinking
	
	Bigger problem for early career researchers - short term post doc contracts etc etc   
	
	
	"despite widespread acknowledgement of the benefits of engaging
	in collaborative research projects, there is a strong feeling amongst members of the
	academic community that collaborative research is not valued as part of an academic
	career within universities. Instead, career progression is considered to rely heavily on
	the quality of the academic’s publication record and their ability to win grant funding
	from competitive, peer-reviewed public sources. Universities need to ensure that their
	recruitment policies and promotion criteria recognise and reward successful commercial
	research collaborations as an integral part of research success in relevant disciplines.51
	If this is already the case, it does not appear that the message is filtering through to
	researchers. This, in turn, affects attitudes towards movement between business and
	academia. The significance attached to REF scores and publication records also acts as a
	barrier for businesspeople seeking to move into academia."  DOWLING REPORT 
	
	Bond report: 68\% of academics surveyed reported that lack of personal time and/or internal resources was a barrier to successful KE activities .
	
	\subsubsection{Brexit}
	I am reluctant to mention the 'B' word but i think no report that looks at the future of Mathematical Research, Industry or Mathematics in Industry can avoid the topic. The whole scientific community is concerned over the future of funding and ability to collaborate (BREXIT LETTER NOBEL LAUREATES) and industry in the UK is facing an uncertain future (REFERENCE OR MORE HERE).
	
	I don't want to spend much time on this, but I would like to say that whatever happens, examples of success stories in industrial mathematics have demonstrated the need for collaboration: with industry, with other fields, with other departments and it will be vital for the continuing success of Industrial Mathematics that the spirit of collaboration is maintained. 
	
	\subsection{Case Study 1: Trip Wire Detection for Land Mines \label{landmines}}
	
	The first case study that really stood out to me, was one brought to the second Industrial Problem-Solving Workshop (a Canadian version of the ESGI) held in Calgary in 1998. I follow an account of the project by one of the attendees, John Stockie \cite{Stockie2015} and the Study Group report \cite{Jessop}.
	
The industrial partner was ITRES Research LTD and they were experimenting mounting a detection camera on a boom ahead of a slowly moving truck which would look vertically downwards. It was hoped that an automatic algorithm would be used to find trip wires appearing in the image. 
	
	A report, the Landmine Monitor, produced in 1999 \cite{landmine} suggests that at the time the of the Study Group there were more than 250million Anti-personnel Mines in Stockpiles of which they were particularly concerned about remotely-delivered, surface laid anti vehicle mines that utilize trip wires which could explode from innocent acts by individuals. 
	
	A clearly defined  problem, that could have huge benefits worldwide so why was this a study group problem? Why hadn't a solution been implemented already? There were some inherent difficulties in detecting a tripwire including:
	\begin{itemize}
	\item wires are often partially covered by foliage
	\item wires are not uniform in illumination 
	\item wires are often purposefully camouflaged and come in a variety of colours and transparencies
	\item other image features may mimic lines such as vegetation 
	\item images are often noisy or blurry as trucks and cameras move or the camera fails to focus. Natural elements also cause additional artefacts in the field of view
	\end{itemize}
	The goal of the week long study group was to have a first attempt at developing an algorithm that was \textsl{robust} enough to cope with the problems above; \textit{reliable }enough to detect trip wires in with near perfect sensitivity and a high specificity and to be \textit{fast} enough to run before the truck detonates a landmine. 
	
	We will look at 3 elements of their work: pre-processing, line detection and improving speed:
	\paragraph{Pre-processing}
\begin{itemize}
	\item \textbf{Laplacian Filter}: Mathematicians love definitions and in this case careful consideration of the definition of a wire in an image indicates a possible direction of work. Defining image intensity to be a function $u(x,y)$ of position $(x,y)$ and thus a line, a sharp edge, is one in which the function $u(x,y) $ has a high curvature.
	To enhance this feature the convolution of the image with a Laplacian filter is taken. This has a similar effect to taking the second derivative and accentuates regions of high curvature.	
	\item \textbf{Edge detection }: Edges are then found in the filtered image using either the Sobel edge detector or the log method. The result is a binary image with found edges indicated. 
	\end{itemize}
	
	\paragraph{Line Detection}
	Continuing the theme of precise definitions: the differentiating factor between any old line and a potential trip wire is that a wire is locally straight and although could be partially hidden consists of a sequence of co-linear line segments. In order to detect such lines they use: 
		\begin{itemize}
		\item \textbf{Radon Transform}: The Radon transform of a 2D image given by:
		
		\begin{equation}
			R(\rho, \theta)=\int_{R^2} u(x,y)\delta(x\cos(\theta)+y\sin(\theta)-\rho) dx dy 					
		\end{equation}
		transforming the image from an $(x,y) $ domain to a $ (\rho, \theta)$ domain. Peaks in the $ (\rho, \theta)$ domain correspond to lines of the form $ x\cos(\theta)+y\sin(\theta)=\rho$ in the $(x,y) $ domain. 
		\item \textbf{Threshold transformed images }: Finding the peaks in the $ (\rho, \theta)$ domain requires some sort of threshold. If the threshold is too high then no wires are found, too low and there are too many false positives. What is defined as a wire is independent of individual images but variations in images can cause additional noise inflating all the values in the $ (\rho, \theta)$ domain causing issues. This is not a simple issue. 
	\end{itemize}
	\paragraph{Algorithm Speed}
		\begin{itemize}
		\item \textbf{Using the FFT}: A the time of the Study Group Matlab's inbuilt Radon transform did not take advantage of the fast Fourier transform to speed up implementation. Such a possibility was discussed. 
		\item \textbf{Exploiting the method of image acquisition}: Images are received from the camera as a sequence of lines that are constantly updated. Discussion was had as to how the intensities in the $ (\rho, \theta)$ domain can be continuously updated as each new line is attained, reducing the number of computations compared to repeatedly taking the radon transform of the whole image. 
	\end{itemize}
	
	The algorithm produced during the week was in no way perfect, struggling with some of the more 'difficult' images and ones where the wires were oblique. Results were presented to the industry partner but future collaboration was not forthcoming, a shame but potentially due to the military applications of such work. There is evidence that ITRES continued working on this problem with conference proceedings released in 2000 \cite{Babey}. \emph{TO READ THIS PLEASE!!!!!!!!}
	

	
	
	\subsection{Case Study 2 - Shelter: Homeless Populations \label{homeless}}
	
To READ https://www.gov.scot/binaries/content/documents/govscot/publications/research-publication/2018/06/health-homelessness-scotland/documents/00536908-pdf/00536908-pdf/govscot\%3Adocument
	
   Brought to ESGI 29, held in Oxford in March 1996 a problem from the charity Shelter to model the numbers of homeless and non-homeless people in a Borough. The model should then be able to predict changes in homeless populations as a result in changes of local policy. I follow the report produced at the end of the study group \cite{Shelter1996}
	
	The populations was split into broad classes: 
	\begin{itemize}
		\item $ T= $ N umber of households in temporary accommodation (Homeless)
		\item $  P= $ Number of households permanently resident in council housing stock
		\item $ G= $ Number of households in private sector accommodation 
	\end{itemize}
	This is then further subdivided as:
	\begin{itemize}
		\item $ P_R= $ Number of households in council stock and not  seeking transfer to other council housing stock 
		\item $ P_N= $ Number of households in council stock and seeking transfer 
		\item $ G_R= $ Number of households in private sector accommodation seeking transfer to council housing stock 
		\item $ G_N= $ Number of households in private sector accommodation and not seeking transfer
	\end{itemize}

Hence we have the following relations 
\begin{itemize}
	\item $ P=P_R+P_N $
	\item $ G=G_R+G_N $
	
\end{itemize}

Also define: 
\begin{itemize}
	\item Number of Households on the register,$  R=T+P_R+G_R $
	\item $ P_0 = $ Total availability of housing stock 
	\item $ G_0= $ Total number of households in the Borough
\end{itemize}

The flows in and out of population groups are shown in Figure \ref{fig:homelessrates}. A few assumptions about negligible flows have been assumed to simplify the problem: it is assumed that homeless families only come form the private sector and also that flows from $ G_R $ to $ G_N $ and $ T $ back to $ G $ are negligible. 

This gives a system of differential equations
\begin{eqnarray}
\frac{dG_N}{dt}=-k_5 G_N-k_3 G_n +k_6 P\\
\frac{dG_R}{dt}=-k_5 G_R +k_3 G_N -k_4 (P_0-P)G_R\\
\frac{dT}{dt}=k_5 G -k_1(P_0 -P)T\\
\frac{dP_N}{dt}=-k_6 P_N -k_7 P_N +(P_0-P)(k_4 G_R +k_1 T+k_8 P_R)\\
\frac{dP_R}{dt}=-k_6 P_R+k_7 P_N -k_8(P_0-P)P_R
\end{eqnarray}
	 \begin{figure}
	 	\centering
	 	\includegraphics[width=0.9\linewidth]{Report_images/homeless_rates}
	 	
	 	\caption{A diagram to show  flows and and associated rates of households moving  between population groups}
	 	\label{fig:homelessrates}
	 \end{figure}
	 
	These equations implicitly assume that:  
	\begin{itemize}
		\item Birth and death rates can be neglected
		\item No migration in and out of the borough 
		\item Rates depend only on the present circumstances there is no delay e.g. due to administrative processes 
	\end{itemize}

These equations look intractable but substituting $ G_0=P+T+G_r+G_N $ and $ G_0=G+P+T$ reduces the system to 3 differential equations:

\begin{eqnarray}
\frac{dT}{dt}=k_5 G -k_1(P_0 -P)T\\
\frac{dG_R}{dt}=-k_5 G_R+k_3(G_0-P-T-G_R)-k_4(P_0-P)G_R\\
\frac{dP}{dt}=-k_6P+(P_0-P)(k_4G_R+k_1T)
\end{eqnarray}
	
	They looked for equilibrium solutions, so setting the left hand side of the equations to zero. They were able to see that a feasible equilibrium solution always exists, where the number of occupied council houses is positive but no greater than the number actually available. 
	
	Looking for analytic solutions of the equations in full generality was deemed unproductive for the short intensive time available in a Study group so focus was moved to looking for numerical solutions. 
	
	Initial values were chosen to represent a typical metropolitan borough. calculations were carried out with numbers of people rather than families. They found that the steady state solution was locally stable but that the values showed high sensitivity from the initial data and values. Thus generalised discussion about results for the "typical" borough or for all boroughs is not possible. They did however manage to make some useful inferences: 
	

	\begin{itemize}
		\item The populations only settle down to the steady state values over a period of about 30 years. This is a much larger time frame than changes in local and national policies and is large enough that births, deaths and migration should be taken into account. Future work should either aim to increase the complexity of the model to take this into account or should look closer at the initial dynamics and variation. 
		\item Reducing the constant $ k_1 $ by a factor of 10, signifying a change in policy so that much less priority is giving to the homeless on council waiting lists causes very little change apart from a 10-fold increase in the numbers of homeless in the borough. The amount of vacant council property increased a little but not sufficiently to cope with the increase in homeless families. The factor $ k_1 $ is important. 
	\end{itemize}
	
	The work was continued after the study group, including a paper produced based mainly on the work done during the study group \cite{Byatt-Smith2003}. The work was also continued as the MSc project and then the PhD project of Andrew Waugh, supervised by one of the study group attendees Andrew Lacey, at Herriot Watt University. 
	
	We look at one of the extensions he made to the Study Group Model in the form of a points based model \cite{Waugh1999}. This aims to model the mechanisms used to allocate houses to those on the waiting list. 
	
	
	
	\subsection{Case Study 3: Optimisation of the “118” Emergency Management System in Milan \label{Milan} }
	The book "European Success Stories in Industrial Mathematics" \ref{Lery2011} contains a wealth of great examples of effective knowledge transfer and collaboration. One particular example, "Optimisation of the "118" Emergency Management System in Milan" attracted me because of its real-world importance and clear mathematical optimisation problem but also because of a particular sentence: 
	
	\begin{quote}
		''In spite of being developed by academic personnel, the outcome of the project has been actually implemented''
	\end{quote}

	Perhaps, not a sentence I should have taken out of context, but CONTINUE HERE 
	
	

	I follow the 2013 paper by R. Arinhieri, G. Carello and D. Morale \cite{R.AringhieriG.Carello2013}  to describe some fo the work behind this case study.
	
	They focused on 3 areas: evaluation of the current EMS system; study of operational policies which can improve the system performance through a simulation model and using optimisation to fund an alternative set of ambulance posts. 
	\begin{figure}
		\centering
		\includegraphics[width=0.9\linewidth]{Report_images/MilanEMS}
		\caption{The process of responding to an emergency call}
		\label{fig:milanems}
	\end{figure}
\paragraph{Evaluation of the current EMS system }
	Figure \ref{fig:milanems} documents  the process of responding to an emergency call. A few points to highlight: 
	\begin{itemize}
		\item Italian law states that urgent calls have to be responded to in 8 minutes in urban areas. This threshold is also a major measure of performance 
		\item The Milano EMS used two types of ambulance: 
		\begin{itemize}
			\item Standard/prepaid ambulances - a set composed of 29 ambulances that are always available. They represent a fixed cost regardless of the number of missions performed. Ambulances are located at ambulance posts ready to be deployed.
			\item "Volunteer" ambulances - run by various volunteer organisations who own them. They can be summoned if needed and the cost is per mission. They are based at their organisation's head quarters.
		\end{itemize}
		The focus of the work was on the first set: to try and serve the majority of the requests with standard ambulances in order to reduce costs and ensure consistency of quality of care. 
		\item After the ambulance has been assigned to the point were it returns to its post or depot after completing a mission the ambulance is unavailable and cannot be diverted. 
	\end{itemize}
\paragraph{Development of a simulation model in order to test operational policies }
Analysis of the historical data found that there was room for improvement- on average 60.1\% of the urgent calls meet the 8 minute response time with a 95\% confidence interval of $ [56.13\%, 64.06\%] $. Any investment in infrastructure to try and improve these figures would be expensive and with no guarantee of success: a mathematical model could help test different scenarios. 

In the model ambulance locations are plotted on a map of Milan at each point in time, the speed assigned to each ambulance is a function of the time of the day and the area in which the ambulance is currently located. This is more  computationally intensive compared to an event based model where the movement of tan ambulance from a place to another one is usually represented by an new event, the actual movement of the ambulance is not a part of simulation model.The model focused on just the set of standard ambulances.

Each emergency request was generated by using real data of a given day, a variety of days were selected to test the model, including some selected critical days which could be representative of emergency scenarios. 

The model was initialised to model current procedure and for the 7 days worth of events used, the model predicted a that 63.92\% of the urgent calls would be responded to within the 8minute deadline. This is within the 95\% confidence interval produced from historical data and suggests the simulation is representative of actual EMS behaviour. 

Various parameters could now be varied in order to test different improvement strategies:
\begin{itemize}
	\item \textbf{Increases to  ambulance average speed}: Road improvements, reserved lanes and systems to wave ambulances through traffic lights could all increase average ambulance speed. An average increase of 5km/h was shown in the model to increase the number of urgent calls reached within 8minutes by approximately 20\%.
	\item \textbf{Adding a new ambulance}:  Adding a new standard ambulance to the fleet would be an expensive investment and the model showed that with the current set of ambulance posts that there would be little improvement in response rates, due to ambulances gathering in a few posts and leading to unbalanced global coverage. 
	\item \textbf{Moving to "smart" ambulances}: The possibility of summoning an ambulance before it reaches its post on return from a mission, or in redirecting an ambulance on its way to a less urgent mission can be evaluated using the model, because of the agent based approach to modelling ambulances. At the time of the work, ambulances were not equipped with GPS systems and therefore this sort of assignment was not possible. On the heaviest days the model was predicting up to a 20\% improvement in the number of urgent calls responded to within the 8 minute deadline, similar figures to that of increases in ambulance speed but at potentially a much reduced cost and less dependent on external factors such as weather. 
\end{itemize}

\paragraph{Finding an alternative set of ambulance posts}
HELP!!!! NEEDS MORE OF MY OWN INTUITION 

Low Priority Calls Coverage (LPCC) optimisation model was developed by the authors to take a potential set of posts and solve a minimisation problem, to discover the theoretical minimum number of required ambulances and where their posts should be located. First some definitions of variables:
\begin{itemize}
	\item $V=$ the set of points to be covered. For each point $ i \in V $, $d_i^h$ denotes the amount of hight priority demands and $d_i^l$ low priority demands 
	\item $ W= $ the set of candidate post locations. The capacity associated to each post is denoted by $ k_j $.
	\item Let $ W_i^h $ be the set of candidate posts from which a demand point $ i $ can be reached within the 8 minute threshold. 
	We introduce a second time limit in which a certain percentage of non urgent calls must be seen, and similarly let $ W_i^l $ be the set of posts from which $ i $ can be reached in this longer time frame
	\item Let $ y_{ij} $ be the fraction of emergency demand of point $ i  $ served by an ambulance at post $ j $. Similarly $ w_{ij} $ to be the fraction of non urgent demand at point $ i  $ served by an ambulance at point $ j $.
	\item An integer variable $ x_j $ is define for each post $ j $, representing the number of ambulances assigned to the post, this can be zero.
\end{itemize}

\begin{eqnarray}
\min \ z=\sum_{j\in W} x_j \label{1a}\\
s.t.\ &\sum_{j\in W_i^h} x_j\geq1\ \ &\forall i \in V\label{1b}\\
&\sum_{j\in W_i^h} y_{ij}=1\ \ &\forall i \in V\label{1c}\\
&\sum_{j\in W}w_{ij}=1 \ \ & \forall i\in V \label{1d}\\
&\sum_{i \in V} \sum_{j \in W_i^l} d_i^l w_{ij} \geq q \sum_{i \in V} d_i^l\label{1e}\\
&\sum_{i \in V} (d_i^h y_{ij} +d_i^l w_{ij}) \leq k_jx_j & \forall j \in W\label{1f}\\
&x_j\in Z_+, \ w_{ij} \in [0,1], \ &\forall i \in V,\  \forall j \in W \label{1g}
\end{eqnarray}
Where Eq \ref{1a} is the objective function: to minimise the number of required ambulances. Eq \ref{1b} ensures that there is coverage over all the city. Eq \ref{1c} ensures that urgent calls are served within the given time limit. Eq \ref{1d} guarantees that all the low priority calls can be served by and ambulance at any posts. Eq \ref{1e} forces that a given percentage of the non-urgent calls ate served with a second time limit. Eq \ref{1f} limits the number of missions in a given time frames. Finally \ref{1g} defines the domains of the variables 
 
 
For the solution of Milan the city was divided into 493 grid squares, chosen such that the area in each gird is covered by the same set of candidate post locations. Thus as long as each sub area is covered, all possible areas will also be covered. They set $ V=W $, $ q=0.5 $ and the non-urgent time frame to be 30minutes. This gives an optimal solution of 25 ambulances.

The remaining 4 ambulances are then allocated using an iterative procedure where the model from the previous section is used, and ambulance posts given  a ranking determining their utilisation. New posts are added in such a way as to decrease the highest utilisation value , this is iterated until there is a post located for each ambulance. 

This could also be used to design a set of posts to reach a threshold level of performance
	\subsection{Case Study 4: Mathematical Modelling of the Dynamics of Meningococcal Meningitis in Africa \label{Africa }}
The book "UK Success stories in Industrial Mathematics" \cite{Aston2016} is a collection of cases studies selected from those Impact Case Studies submitted to the 2014 Research Excellence Framework. They are seen as shining examples of the mathematical sciences community engaging with problems and organisations outside of academia. 36 problems were selected out of the 250 submitted for the REF and I aimed to pick just one to write up as a case study- a challenge indeed!


The African meningitis belt spans sub Saharan Africa and see periodic fluctuations of meningococcal meningitis, cases appearing every dry season, drop off over the rainy season and a major epidemic emerges every 6-14years. Across the world, meningitis causes about 135,000 deaths annually and substantial data is available across the African meningitis belt, so mathematical modelling of epidemiology of meningococcal meningitis could lead to substantial benefits. Indeed, modelling has been attempted by a number of groups but we follow the work by K.B.Blyuss at the University of Sussex described in "UK Success stories in Industrial Mathematics." and the associated paper \cite{Irving2012}.


\paragraph{Initial Modelling and Identification of Reproduction Number  }
Looking for a simple model the team looked towards the standard SIR model. In the case of meningitis, individuals can be carriers without developing the infection and there is a high ratio of carriers to cases, thus an additional "carrier" class was a necessary addition, giving 4 classes: 
\begin{itemize}
	\item S= Susceptible
	\item C=Carriers
	\item I=Infected
	\item R=Recovered
\end{itemize}
The total population is $ N=S+C+I+R $. The question of temporary immunity was a major part of this work and they considered a variety of models where there was:
\begin{itemize}
	\item no immunity, and thus no recovered class, recovered individuals went straight  to susceptible 
	\item   immunity only if they have had developed a meningitis infection, if individuals had only been a carrier then they got new immunity
	\item immunity if they had been a carrier, developed the infection or both
\end{itemize}

\begin{figure}
	\centering
	\includegraphics[width=0.9\linewidth]{Report_images/meningitis_model}
	\caption{A diagram to show the flows between population groups in the chosen meningitus model, taken from Fig 2 in paper  \cite{Irving2012}}
	\label{fig:meningitismodel}
\end{figure}



 Modelling all 3 cases they found the best fit to the observed data was the last case and the model shown in Figure \ref{fig:meningitismodel}, with rates defined as followed:
\begin{itemize}
	\item $ \beta =$ the transmission rate
	\item $ a= $ the rate at which carriers develop an invasive disease
	\item $ \alpha =$ the rate at which carriers recover 
	\item $ \rho= $ the rate at which individuals with invasive disease recover
	\item $ \phi = $ the rate at which recovered individuals loose their immunity becoming susceptible again 
	\item $ \mu = $ natural death rate
	\item $ \gamma = $ disease induced death rate

\end{itemize}

The birth rate is chosen to be $ b=\mu N +\gamma I  $ so that the total population $ N=S+C+I+R $ is constant. This assumption is only suitable for relatively short term modelling before the assumptions break down but it is a good starting point and gives equations: 

\begin{eqnarray}
	\frac{dS}{dt}= b +\phi R-\beta \frac{S(C+I)}{N}-\mu S\\
	\frac{dC}{dt}=\beta \frac{S(C+I)}{N}-(a+\alpha+\mu)C\\
	\frac{dI}{dt}=aC-(\rho +\gamma+\mu)I\\
	\frac{dR}{dt}=\rho I +\alpha C-(\phi+\mu)R
\end{eqnarray}
	
	Which under rescaling by $ N $ and using  $ I=S+C+I+R $ to remove $ S $ from the equations gives
	\begin{eqnarray}
	\dot{C}=\beta(1-C-I-R)(C+I)-(a+\alpha+\mu)C\\
	\dot{I}=aC-(\rho+\gamma+\mu)I\\
	\dot{R}=\rho I+\alpha C-(\phi+\mu)R
	\end{eqnarray}
	
	We look for steady state solutions when $ \dot{C}=\dot{I}=\dot{R}=0 $. Messy calculations but one finds:
	\begin{eqnarray}
	C^*=\lambda (\phi+\mu)(\rho+\gamma+\mu)\\
	I^*=\lambda a(\phi+\mu)\\
	R^*=\lambda(\alpha(\rho+\gamma+\mu)+\rho\alpha)
	\end{eqnarray}
	
	Where $ \lambda $ is a proportionality constant that can be found by subbing into $  \dot{C}=0 $ giving $ \lambda=0 $ or:
	
	\begin{eqnarray}
	\lambda=\frac{\beta(\gamma+\rho+\mu+a)-(\gamma+\rho+\mu)(\alpha+a+\mu)}{\beta(\rho+\mu+a)[(\rho+\gamma+\mu)(\phi+\mu+\alpha)+a(\phi+\mu+\rho)]}\\
	=\frac{(\rho+\gamma+\mu)(a+\alpha+\mu)}{\beta(\rho+\mu+a)[(\rho+\gamma+\mu)(\phi+\mu+\alpha)+a(\phi+\mu+\rho)]}\frac{\beta(\gamma+\rho+\mu+a)-(\gamma+\rho+\mu)(\alpha+a+\mu)}{(\rho+\gamma+\mu)(a+\alpha+\mu)}\\
	=\frac{(\rho+\gamma+\mu)(a+\alpha+\mu)}{\beta(\rho+\mu+a)[(\rho+\gamma+\mu)(\phi+\mu+\alpha)+a(\phi+\mu+\rho)]}\left( \frac{\beta(\gamma+\rho+\mu+a)}{(\rho+\gamma+\mu)(a+\alpha+\mu)}-1\right) 
	\end{eqnarray}
	
	
	An important quantity to identify when looking at epidemiology models is the reproduction number,  $  R_0 $. As defined in \cite{VanDenDriessche} it is such that if $ R_0 < 1 $, then the disease free  steady state $ (C,I,R)=(0,0,0) $ is stable, and the disease cannot invade the population, but if $  R_0 > 1 $, the disease free steady-state is unstable and an epidemic can take hold. 
	For $ R_0>1 $ a new stable stationary point should exist for values of $ C,I,R>0 $.
	In our case looking at our two potential steady states and the equations for $\lambda$ we find that in order for $C^*, I^*, R^*>0$ we require: 
	
	\begin{eqnarray}
	R_0=\frac{\beta(\gamma+\rho+\mu+a)}{(\gamma+\rho+\mu)(a+\alpha+\mu )}>1
	\end{eqnarray}
	Giving us the reproduction number and valuable information about in which circumstances meningitis will spread or be contained. 
	
Again using \cite{VanDenDriessche}  a more qualitative description fo the reproduction number is  the number of new infections produced by a typical infective individual in a population at a disease free steady state.  If $ R_0 < 1 $, then on average an infected individual produces less than one new infected	individual over the course of its infectious period, and the infection cannot grow. Conversely, if 	$ R_0 > 1 $, then each infected individual produces, on average, more than one new infection, and the disease can invade the population. 


We consider adding a new carrier to an otherwise healthy population. Two things can happen, the carrier can directly infect healthy members of the population or the carrier can become infected and then infect healthy members of the population. 

The expected amount of time the carrier remains a carrier is
 \begin{eqnarray}
\frac{1}{a+\alpha+\mu}
\end{eqnarray}
and as a carrier they infect members of the general population at rate $ \beta $ giving expected number of people infected by a single carrier that doesn't become infected is:
 \begin{eqnarray}
\frac{\beta}{a+\alpha+\mu}
\end{eqnarray}

The carrier becomes infected with probability 
 \begin{eqnarray}
\frac{a}{a+\alpha+\mu}
\end{eqnarray}
calculated from the expect time the carrier remains a carrier multiplied by the rate of contracting the infection. The expected time they remain infected is: 
 \begin{eqnarray}
\frac{1}{\rho+\delta+\mu }
\end{eqnarray}
and they infect healthy individuals with rate $\beta  $

Thus the expected number of new cases is: 

\begin{eqnarray}
R_0=\frac{\beta}{a+\alpha+\mu}+\frac{a}{a+\alpha+\mu}\frac{\beta}{\rho+\delta+\mu }\\
=\frac{\beta(\gamma+\rho+\mu+a)}{(\gamma+\rho+\mu)(a+\alpha+\mu )}
\end{eqnarray}
Matching the result from before.

\paragraph{Adding the seasonal effects  }


To try and explain the seasonal variations in the number of cases, they introduced a seasonally varying rate of disease transmission:
\begin{eqnarray}
\beta(t)=\beta_0(1+\epsilon_\beta \cos(2\pi t))
\end{eqnarray}
	
The model then exhibits a wide variety of dynamical behaviours and importantly, sees chaotic behaviour for a range of values of $\phi$ which is significant as $\frac{1}{\phi}$ is the expected duration of temporary immunity. Simulations suggest that that the model is able to produce the regular annual epidemics as well as epidemics with larger periods, can produce different amplitudes of epidemics and chaotic behaviour such as which is seen in the data. The longer inter epidemic periods, such as the 6-14years seen in the data corresponds to $\frac{1}{\phi}\approx 2 $ years.  
\paragraph{Impact}

\begin{itemize}
	\item The model highlighted the important role of the temporary immunity and the value of the constant $\phi$. A focus on computing an accurate value for this constant or working clinically to alter its value could be hugely beneficial
	\item An accurate value of the reproduction rate, $ R_0 $ is important when optimising vaccination deployment, reducing costs and increasing effectiveness. 
	\item Future work could look at including spatial effects, potentially including satellite and meteorological data. An advance disease warning system, for the larger deadlier outbreaks every 6-14 years would be hugely beneficial. 
\end{itemize}
	\pagebreak
	\section{Extended Case Study - High Intensity Focused Ultrasound}
	
	High Intensity Focused Ultrasound (HIFU) is a non-invasive technique to enhance biological therapies by exposing tissues to acoustic energy. Applications include:
	\begin{itemize}
		\item Destroying cancerous tissue using thermal hyperthermia 
		\item Localised drug delivery 
		\item Functional/structural modification of tissues
	\end{itemize}
In addition paediatric.foetal application are beginning to be explored due to the potential to deliver a non-ionizing energy based therapy,, in a non-invasive manner 
	\begin{itemize}
		\item Transducer produces ultrasound signal focused on a region, $8mm\times2mm\times2mm$
		\item Pressure variations due to the ultrasound lead to a temperature source $Q(\textbf{x},t)$
		\item Temperature $T(\textbf{x},t)$ changes due to the action of the source, diffusion and perfusion due to blood flow. 
		\item High temperatures over a sustained time lead to tissue damage. 
	\end{itemize}
	
	The problem can be split into three areas: focusing the ultrasound signal onto the region, controlling the temperature levels and using the correct amount of heat for the right amount of time to lead to the desired effects and imaging the whole procedure. 
	
	
	\subsection{Set Up}
	Clinical adoption of HIFU has expanded rapidly in recent years partly due to better visualisation and thermometry tools. A typical set up is shown in Figure \ref{fig:howitworks}
			\begin{figure}[H]
		\centering
		\begin{center}$
			\begin{array}{ccc}
			\includegraphics[width=0.33\linewidth]{"Report_images/howitworks"} &
			\includegraphics[width=0.33\linewidth]{"Report_images/howitworks2"}&
			\includegraphics[width=0.33\linewidth]{"Report_images/howitworks3"}
			\end{array}$
		\end{center}
		
		\caption{The set-up of MRI with integrated HIFU, REFERENCE....   }
		\label{fig:howitworks}
	\end{figure}
	\subsection{Imaging}
	Safety and efficacy of the treatment require accurate temperature measurement throughout the thermal procedure.
	
	The method has shown some success in uterine fibroids, prostate cancer and liver tumours but application of the method for treatment of brain tissue remains a challenge due to strong aberration due to the skull bone, defocuses the beam and results in a loss of acoustic pressure. One approach to correct for this is to use imagery to detect the location of the beam and then to correct for the defocusing to refocus the beam. 
	
	MRI thermometry, is a method where a traditional MRI scanner can be used to measure temperature. Temperature measurement is based on the water proton resonance frequency (PRF) shift induced phased differences between dynamic frames. MR dynamic phase images a relative frequency shift is calculated. 
	
	\begin{equation}
	\Delta T= \frac{\Delta \phi}{2 \pi \alpha \gamma B_0 T_E}
	\end{equation}
	
	Where: 
	\begin{itemize}
		\item $\Delta \Phi $ = phase shift 
		\item $\alpha$ = Temperature sensitivity coefficient 
		\item $\gamma$ = gyromagnetic ratio
		\item $B_0$ = magnetic field strength
		\item $T_E$ = echo time 
	\end{itemize}

	Temperature maps are calculated in real time and displayed as overlays on the magnitude image. This can be seen in Figure \ref{fig:mriimage}.
	
	\begin{figure}
		\centering
		\includegraphics[width=0.7\linewidth]{Report_images/MRIimage}
		\caption{Temperature maps are calculated in real time and displayed as overlays on the magnitude image}
		\label{fig:mriimage}
	\end{figure}
	
	
	 paper \cite{Rieke2008}
	
	Drawbacks to this is that temperature in bone and fat tissue cannot be measured with the PRF method and the phased of a MR image is sensitive to disturbances such as transducer movement and magnetic field drift and to patient movement.
	
	
	

	
	\subsection{Focusing the ultrasound signal}
	
		MAGNETIC RESONANCE ACOUSTIC RADIATION FORCE IMAGING 	\cite{Hosseini2018}
		
	
	\subsection{Modelling Heat transfer and thermal dose}
	
	Different temperatures during heating and cooling, diffusion from the target area and possible patient movement all contribute to uneven heating in different areas. Some sort of normalisation is required to calculate the effect on each individual cell. Some sort of thermal dose measure is required, from which effective dose and dangerous levels can be estimated. 
	
	Proposed in the 1940s, the Pennes Bioheat Transfer Model, \cite{Pennes1948} models heat transfer in the body die to externally applied heating/cooling sources. It has three terms, a diffusion term, an external heating term, and a convection term due to the perfusion of heat due to blood flow. The model accounts for the thermal conductivity, specific heat capacity and blood perfusion of specific tissue types. 
	\begin{equation}
		\rho c \frac{\partial T}{\partial t}= \nabla\cdot k \nabla T-\omega c_b (T-T_b)+Q
	\end{equation}
	where:
	
	\begin{itemize}
		\item $ \rho = $ Tissue density[$ kg/m^3 $]
		\item  $ c= $ Specific heat capacity [$ J/kg/\degree C  $]
		\item $ k= $ Thermal conductivity  [$ W/m/\degree C $]
		\item $ \omega = $ Blood perfusion [$ kg/m^3/s $]
		\item $ Q= $ Heat deposition from ultrasound $ [W/m^3 $]
		\item $ T_b= $ Arterial blood temperature [$ 37\degree C $]
		\item $ t= $ Time [$ s $]
	\end{itemize}
	
	Values for specific tissue types are shown in Figure \ref{fig:tissueproperties}. 
	
	\begin{figure}
		\centering
		\includegraphics[width=1\linewidth]{Report_images/tissueproperties}
		\caption{Material properties of tissues, NEEEEEEEEEEEEEDDDDDDDD A REEEEEEEEEFFFFFFFFFEEEEEEEERRRRRRRRREEEEEEEEEEENNNNNNCCCCCCCCCCCEEEEEEEEEEEE}
		\label{fig:tissueproperties}
	\end{figure}
	
	Initially, Harry Pennes experimented on patients by inserting thermocouples into patients' forearms but the model has been validated over the years against other models and is applicable to many different heating sources and types.
	
	
	
	
	Classical theory \cite{Sapareto1984}  measures thermal dose as a time integral over treatment:
	
\begin{equation}
TD(t)=\int_{0}^{t}r^{43-T(t)}dt,\ \ \text{where  } r=
\begin{cases}
0.25, & \text{if}\ T\leq 43 \degree C \\
0.5, & \text{if}\ T>43 \degree C
\end{cases}
\end{equation}
	
	The value taken as sufficient "dose" to cause thermal necrosis in soft tissue is 240 EM (equivalent minutes) at $43 \degree C $. One can also calculate that it takes just a 1 second exposure at $57 \degree C $ to give a thermal dose of 273 EM and thermal necrosis. For every degree above  $43 \degree C $ the required time to coagulate the tissue halves. 
	
	The model has its advantages for HIFU: importantly it is valid for the high temperatures seen in HIFU and it is valid for tissues with different thermal sensitivity although the threshold for thermal dose required for cell death changes. 
	
	However, the model has its draw backs as there is a non linear response between temperature and cell death - higher probability of dying with increasing temperature and time exposure. Also, measuring dose does not directly predict damage. Finally, tissues have varying thermal sensitivity and will ablate at different thermal doses which will add extra complexity to the model.
	
		An alternative model for cell damage was introduced in  paper  \cite{Zhou2007}. Zhou et al suggest that damage is the result of a chemical reaction 
		\begin{equation}
		P\rightarrow D
		\end{equation}
		
		Where P represents a healthy folded protein and D a denatured protein. The irreversible process of protein denaturization occurs at an Arrhenuius reaction rate:
		
		\begin{equation}
			r(T)=A\exp\left( \frac{E}{RT}\right) 
		\end{equation}	
		where A is a frequency factor, E the energy of activation of the denaturation reaction, R is the universal gas constant and T is the absolute temperature of the tissue. From this a damage fraction can be calculated:
		\begin{equation}
			\Omega(t)=\log\left( \frac{P_0}{P(t)}\right) =\int_{0}^{t}A\exp\left( \frac{E}{RT}\right) dt
		\end{equation}
		where $ P_0$ is the initial amount of folded protein.
		
		The idea is that a value $\Omega_\theta$, a damage threshold, is found such that $\Omega>\Omega_\theta$ then irreparable damage has occurred. In the study group a value $\Omega_\theta=0.63$ is chosen for muscle. 
		
		
	 	The parameters, A, E and $\Omega_\theta$, the damage threshold, can be varied to match different tissue types. Originally developed for use to model damage from a laser perhaps more work needed....not sure -MORE HEREEEEEEEEEEEEEEEEEEEEEEEEEE!!!!!!!!!!!!!!
	 	
	 	
	 	\subsubsection{Study group modelling}
	 	
	 	During the study group they focused on looking at muscle tissue where the approximate values are:
	 	\begin{itemize}
	 		\item $\rho c \approx 3 \times 10^6$
	 		\item $\gamma \approx 2000$
	 		\item $Q \approx 4\rho c-10 \rho c$
	 		\item length scale of $L \approx 10^{-3}$
	 		\item $k \approx 1/2$
	 	\end{itemize}
 	
 	hence near the focus we have 
 	\begin{itemize}
 		\item $\nabla (k\nabla T) \approx 10^6$
 		\item $ \gamma (T-T_b) \approx 10^3$
 		\item $ Q \approx 10^6$
 	\end{itemize}
 and hence we can conclude that at the focus, perfusion is unimportant unless we are close to a major blood vessel.  Q diminishes rapidly away from the focus, and so both perfusion and diffusion act together to reduce the temperature. 
 
 They chose to rewrite the Pennes equation as a radially symmetric system close to the centre. The used the approximation: 
 \begin{equation}
 Q=Q_0 \rho c \left(\frac{J_1\left(\frac{r}{L}\right)}{\frac{r}{L}}\right)^2= Q_0\rho c \left(\frac{J_1(s)}{s}\right)^2
 \end{equation}
 where $s=r/L$ and this gives the equation:
 \begin{equation}
 T_t=\frac{1}{6s}(sT)_ss-\frac{2}{3}\times 10^{-3}(T-T_b)+Q_0 \left(\frac{J_1(s)}{s}\right)^2
 \end{equation}
 with initial conditions $$ T_s(0)=0 , \ \ \ T(\infty)=T_b .$$
 
 This can be solved numerically with the Crank Nicholson method, with rapid converge utilising the sparsity of the resulting matrices. 
 
 They ran with $Q_0=5$ for $0<t<30$ and then for $Q_0=0$ for $t>30$. Reproducing this method I get the result shown in Figure 
 
 \begin{figure}
 	\centering
 	\includegraphics[width=0.7\linewidth]{Report_images/studygroupcode}
 	\caption{A reproduction of the graph produced during the study group of temperature profile against time, where there is heating for the first 30sec before the tissue is left to cool}
 	\label{fig:studygroupcode}
 \end{figure}
 
\subsection{Current Work } 
 \subsubsection{Aims}
 
\begin{itemize}
	\item  Work on modelling heat transfer which encompasses: 
\begin{itemize}
	\item  Thermal conductivity
\item  Thermal diffusion 
\item  Specific heat capacity
\item  Perfusion 
\end{itemize}
 of the tissue of interest but also including its surrounding structures primarily bone in our case.
 
\item To use this  to model thermal dose and thermal damage of the tissues and surrounding structures. 
\end{itemize}
 
 \subsubsection{2D model homogeneous muscle tissue}
Initially, I looked at creating a 2D model of the set up modelled in the study group. Thus thermal conductivity, thermal diffusion, specific heat capacity, perfusion constants are taken for muscle 
Q was modelled as a Gaussian at the centre of the area of interest and a constant temperature of 37 degrees  were taken as Dirichlet boundary conditions 

To get some feeling for the equations initially I discretised the problem using a central difference in space and a forward difference in time. Thus if $T^i_{m,n}= T(i\delta t, m\delta x, n\delta y)$ we have that:

\begin{eqnarray}
\rho c \left( \frac{T^{i+1}_{m,n}-T^i_{m,n}}{\delta t}\right)= k\left[  \frac{(T^i_{m+1,n}-2T^i_{m,n}+T^i_{m-1,n})}{(\delta x)^2}+\frac{(T^i_{m,n+1}-2T^i_{m,n}+T^i_{m,n-1})}{(\delta y)^2}\right] +Q^i_{m,n}-\omega c_b(T^i_{m,n}-T_b) \label{euler method}
\end{eqnarray}

This is an explicit method and so easy to implement and indeed the images produced, shown in figure \ref{fig:forwardcentraldifference} seem to make sense from what we expect physically. 

\begin{figure}
	\centering
	\includegraphics[width=0.9\linewidth]{Report_images/forward_central_difference}
	\caption{Heating modelled using a forward difference in time and central difference in space.}
	\label{fig:forwardcentraldifference}
\end{figure}

Although I am able to make use of some of the numpy linear algebra routines in built into python, making this code relatively quick to run, adding structure by changing the values of the constants depending on location in the array means many of these routines cannot be implemented, increasing code cost. We do note however that due to the large values of $\rho c$ then this method should be stable for a range of values of $\delta t$ and $\delta x$.

\begin{figure}
	\centering
	\includegraphics[width=\linewidth]{"Report_images/2d_homogenous heating and cooling"}
	\caption{2D model of heating and cooling of muscle tissue. The first line is heating, second line cooling.}
	\label{fig:2dhomogenous-heating-and-cooling}
\end{figure}
Consider a 1D version of equation \ref{euler method}:

\begin{eqnarray}
T^{i+1}_m=T^i_m+\frac{k\delta t}{\rho c (\delta x)^2}[T^i_{m-1}-2T^i_m+T^i_{m+1}]+\frac{\delta t}{\rho c}[Q^i_m-\omega c_b (T^i_m-37)]
\end{eqnarray} 

Which can be written in the form $$ T_{\delta x}^{n+1}=A_{\delta x}T_{\delta x}^{n}+f_{\delta x}^{n} $$
where 

$$ A_{\delta x}= \left( \begin{array}{ccccc}
1- 2\frac{k \delta t}{(\delta x)^2 \rho c} -\frac{\delta t \omega c_b}{\rho c }& \frac{k \delta t}{(\delta x)^2 \rho c} & 0 &\cdots  &0  \\ 
\frac{k \delta t}{(\delta x)^2 \rho c}& \ddots  & \ddots  &  & \vdots \\ 
0& \ddots  &  & \ddots  & 0 \\ 
0&  & \ddots  & \ddots  & \frac{k \delta t}{(\delta x)^2 \rho c} \\ 
0& \cdots &  0&\frac{k \delta t}{(\delta x)^2 \rho c}  & 1- 2\frac{k \delta t}{(\delta x)^2 \rho c} -\frac{\delta t \omega c_b}{\rho c }
\end{array}  \right)  $$

This is a  Toeplitz tridiagonal matrix and thus it has eigenvalues:

\begin{eqnarray}
\lambda_k = 1-\frac{\delta t \omega c_b}{\rho c}-4\frac{k \delta t}{(\delta x)^2 \rho c}\sin^2\left( \frac{k\pi}{2d+2}\right) 
\end{eqnarray}

Where $D$ is the matrix dimension and $k=1,...,d$. For stability, as our matrix is symmetric and thus normal, we require that the absolute values of the  eigenvalues are less than one. As the second and third term are both very small due to $\rho c \approx 10^6$. 


\begin{figure}
	\centering
	\includegraphics[width=0.7\linewidth]{Report_images/study_group_code_2D_model}
	\caption{Temperature of a point a short distance from the focus measured thorughout a heating and cooling period}
	\label{fig:studygroupcode2dmodel}
\end{figure}


To reduce computational cost we try a semi-discretisation to give an ODE in time, and then use pythons inbuilt ODE solver, odeint. This is based on the  lsoda from the FORTRAN library odepack which automatically selects between nonstiff (Adams) and stiff (BDF) methods. The ODE is given by:

\begin{eqnarray}
\rho c \frac{dT}{dt}= k\left[  \frac{(T^i_{m+1,n}-2T^i_{m,n}+T^i_{m-1,n})}{(\delta x)^2}+\frac{(T^i_{m,n+1}-2T^i_{m,n}+T^i_{m,n-1})}{(\delta y)^2}\right] +Q^i_{m,n}-\omega c_b(T^i_{m,n}-T_b).
\end{eqnarray}

Ignoring the perfusion term, valid because it is small, we can get models for heating and cooling as given in \ref{fig:2dhomogenous-heating-and-cooling}, again there is nothing unexpected. Additionally, recording temperatures from a small distance away from the focus and plotting a graph of temperature against time gives the result as shown in figure \ref{fig:studygroupcode2dmodel}. This is similar to the temperature profile shown in figure \ref{fig:studygroupcode}. For the moment we ignore exact units (MORE NEEDED!!!!!!!!!!!!)

Adding back in the perfusion term, despite how small it is, has a dramatic effect on the numerical stability (need to prove this). Problems originating at the boundary are particularly common. Examples are shown in figure \ref{fig:instability}. WHHHHHHHHHHHHHYYYYYYYYYYYYYYY????????

 
 
  



\begin{figure}
	\centering
	\includegraphics[width=0.95\linewidth]{Report_images/instability}
	\caption{Examples of instability caused by the perfusion term}
	\label{fig:instability}
\end{figure}

\subsubsection{2D model with added bone structures}	 	

Initially we changed the constants

\begin{itemize}
	\item  Thermal conductivity
	\item  Thermal diffusion 
	\item  Specific heat capacity
	\item  Perfusion 
\end{itemize}

in a particular region of the domain to that for bone. The region can be seen on the left of figure \ref{fig:bone-without-changing-q}. The gaussian source is focused at position $(50,50)$.  This had very little effect, because the increase in the density of bone is offset by a decrease in the specific heat capacity of bone and decrease in thermal conductivity , comparing $\frac{k}{\rho_{muscle} c_{muscle}} = \frac{0.5}{3430 \times 1041} \approx 1/7141260 $ and $\frac{k}{\rho_{bone} c_{bone}} =\frac{0.38}{ 1420 \times 1700} =1/ 6352631$. This is not a huge difference and it would take a long time for significant differences to be visible and running 1000 steps gave a residual of the difference between the bone and no bone simulations on the right of figure \ref{fig:bone-without-changing-q}. We are relieved to see there is a slight lack of symmetry with darker areas suggesting a greater change where the bone lies. However, the relative difference is still very small, less than a degree. 

\begin{figure}
	\centering
	\includegraphics[width=0.7\linewidth]{"Report_images/bone without changing Q"}
	\caption{TO DO PLEASE}
	\label{fig:bone-without-changing-q}
\end{figure}

We double check that the code is working correctly by Code was working correctly by choosing the "bone" to have a diffusivity constant of 10 times smaller than that of muscle. After 1000 steps this gives the results shown in figure \ref{fig:change-diffusion-constant}, which clearly shows a lack of symmetry and a slower diffusion process occurring in bone leading to lower temperatures in the bone. 

\begin{figure}
	\centering
	\includegraphics[width=0.8\linewidth]{"Report_images/change diffusion constant"}
	\caption{Experimenting with changing diffusion constant in a strip lying in the top half of the region, modelling bone}
	\label{fig:change-diffusion-constant}
\end{figure}

This was very unexpected compared to what is seen  experimentally. For example as discussed in paper \cite{Cochard2009}, by Caochard et al. they discuss how thermal ablation induced by high intensity focused ultrasound has produced promising clinical results to treat hepatocarcinoma and other liver tumors. However skin burns have been reported due to the high absorption of ultrasonic energy by the ribs. A very interesting paper that discusses how to correct for the ribs effect on focusing and analysis derived from time reversal mirrors to reduce heating of the ribs. They point to Duam et al. \cite{Daum1999}  in which it was reported that using magnetic resonance temperature monitoring of pigs in vivo that  temperature elevations were five times higher on the ribs than in the intercostal space. 

Looking back to the table in figure \ref{fig:tissueproperties} we see that we have yet to take into account the difference in the attenuation coefficient in bone and in muscle. This would effect Q, the source term in the equation and so we need to model this differently. 

  



In doing this we got....


Code up Crank Nicholson...more usable format for code....


\subsubsection{Modelling thermal dose}
\subsubsection{Units and other stuff}


  
	\section{Final Thoughts} 

\pagebreak	
	\bibliography{Industrial_Reading_Project}
	\bibliographystyle{plain}
\end{document}